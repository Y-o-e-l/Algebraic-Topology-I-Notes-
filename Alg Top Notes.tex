\documentclass[leqno, openany]{memoir}
\setulmarginsandblock{3cm}{3cm}{*}
\setlrmarginsandblock{3cm}{3.5cm}{*}
\checkandfixthelayout

\usepackage{amsmath}
\usepackage{amssymb}
\usepackage{amsthm}
%\usepackage{MnSymbol}
\usepackage{bm}
\usepackage{accents}
\usepackage{mathtools}
\usepackage{tikz}
\usepackage{tikz-cd}
\usetikzlibrary{calc}
\usetikzlibrary{automata,positioning}
\usepackage{tikz-cd}
\usepackage{forest}
\usepackage{braket} 
\usepackage{listings}
\usepackage{mdframed}
\usepackage{verbatim}
\usepackage{physics}
%\usepackage{/home/patrickl/homework/macaulay2}

%font
\usepackage[sc]{mathpazo}
\usepackage{eulervm}
\usepackage[scaled=0.86]{berasans}
\usepackage{inconsolata}
\usepackage{microtype}

%CS packages
\usepackage{algorithmicx}
\usepackage{algpseudocode}
\usepackage{algorithm}

% typeset and bib
\usepackage[english]{babel} 
\usepackage[utf8]{inputenc} 
\usepackage[T1]{fontenc} 
\usepackage[backend=biber, style=alphabetic]{biblatex}
\usepackage[bookmarks, colorlinks, breaklinks]{hyperref} 
\hypersetup{linkcolor=black,citecolor=black,filecolor=black,urlcolor=black}

% other formatting packages
\usepackage{float}
\usepackage{booktabs}
\usepackage{enumitem}
\usepackage{csquotes}
\usepackage{titlesec}
\usepackage{titling}
\usepackage{fancyhdr}
\usepackage{lastpage}
\usepackage{parskip}

\usepackage{lipsum}

% delimiters
\DeclarePairedDelimiter{\gen}{\langle}{\rangle}
\DeclarePairedDelimiter{\floor}{\lfloor}{\rfloor}
\DeclarePairedDelimiter{\ceil}{\lceil}{\rceil}


\newtheorem{thm}{Theorem}[section]
\newtheorem{cor}[thm]{Corollary}
\newtheorem{prop}[thm]{Proposition}
\newtheorem{lem}[thm]{Lemma}
\newtheorem{conj}[thm]{Conjecture}
\newtheorem{quest}[thm]{Question}

\theoremstyle{definition}
\newtheorem{defn}[thm]{Definition}
\newtheorem{defns}[thm]{Definitions}
\newtheorem{con}[thm]{Construction}
\newtheorem{exm}[thm]{Example}
\newtheorem{exms}[thm]{Examples}
\newtheorem{notn}[thm]{Notation}
\newtheorem{notns}[thm]{Notations}
\newtheorem{addm}[thm]{Addendum}
\newtheorem{exer}[thm]{Exercise}

\theoremstyle{remark}
\newtheorem{rmk}[thm]{Remark}
\newtheorem{rmks}[thm]{Remarks}
\newtheorem{warn}[thm]{Warning}
\newtheorem{sch}[thm]{Scholium}


% unnumbered theorems
\theoremstyle{plain}
\newtheorem*{thm*}{Theorem}
\newtheorem*{prop*}{Proposition}
\newtheorem*{lem*}{Lemma}
\newtheorem*{cor*}{Corollary}
\newtheorem*{conj*}{Conjecture}

% unnumbered definitions
\theoremstyle{definition}
\newtheorem*{defn*}{Definition}
\newtheorem*{exer*}{Exercise}
\newtheorem*{defns*}{Definitions}
\newtheorem*{con*}{Construction}
\newtheorem*{exm*}{Example}
\newtheorem*{exms*}{Examples}
\newtheorem*{notn*}{Notation}
\newtheorem*{notns*}{Notations}
\newtheorem*{addm*}{Addendum}


\theoremstyle{remark}
\newtheorem*{rmk*}{Remark}

% shortcuts
\newcommand{\Ima}{\mathrm{Im}}
\newcommand{\A}{\mathbb{A}}
\newcommand{\N}{\mathbb{N}}
\newcommand{\R}{\mathbb{R}}
\newcommand{\C}{\mathbb{C}}
\newcommand{\Z}{\mathbb{Z}}
\newcommand{\F}{\mathbb{F}}
\newcommand{\Q}{\mathbb{Q}}
\renewcommand{\k}{\Bbbk}
\renewcommand{\P}{\mathbb{P}}
\newcommand{\M}{\overline{M}}
\newcommand{\g}{\mathfrak{g}}
\newcommand{\h}{\mathfrak{h}}
\newcommand{\n}{\mathfrak{n}}
\renewcommand{\b}{\mathfrak{b}}
\newcommand{\ep}{\varepsilon}
\newcommand*{\dt}[1]{%
   \accentset{\mbox{\Huge\bfseries .}}{#1}}
\renewcommand{\abstractname}{Official Description}
\newcommand{\mc}[1]{\mathcal{#1}}
\newcommand{\T}{\mathbb{T}}
\newcommand{\mf}[1]{\mathfrak{#1}}
\newcommand{\mr}[1]{\mathrm{#1}}
\newcommand{\ms}[1]{\mathsf{#1}}
\newcommand{\ol}[1]{\overline{#1}}
\newcommand{\wt}[1]{\widetilde{#1}}
\newcommand{\wh}[1]{\widehat{#1}}

\newcommand{\CP}{\mathbb{CP}}
\newcommand{\RP}{\mathbb{RP}}
\newcommand{\powerset}{\mathscr{P}}
\newcommand{\pset}{\mathscr{P}}
\newcommand{\catname}{\mathsf}
\newcommand{\p}{^{\prime}}
\newcommand{\pp}{^{\prime\prime}}
\newcommand{\iso}{\cong}
\newcommand{\eps}{\varepsilon}
\newcommand{\dx}{\,dx}
\newcommand{\dy}{\,dy}
\newcommand{\dz}{\,dz}
\newcommand{\de}{\,d}
\newcommand{\dvol}{d\!\vol}
\newcommand{\ceq}{\coloneqq}
\newcommand{\tto}{\rightrightarrows}
\newcommand{\nat}{\Rightarrow}
\newcommand{\PGL}{PGL}
\newcommand{\PSL}{PSL}
\newcommand{\SL}{SL}
\newcommand{\GL}{GL}
\newcommand{\qd}{\ \ \ \ \ \ \qedhere}
\newcommand{\proj}{\mathbb{P}}
\newcommand{\cL}{\mathcal{L}}
\newcommand{\sopin}{\Delta^{\text{op}}_{\text{inj}}}
\newcommand{\opp}{^\text{op}}

\newcommand{\onto}{\twoheadrightarrow}
\newcommand{\into}{\hookrightarrow}

\DeclareMathOperator{\Grp}{\mathsf{Grp}}
\DeclareMathOperator{\Ab}{\mathsf{Ab}}
\DeclareMathOperator{\Ring}{\mathsf{Ring}}
\DeclareMathOperator{\Mod}{\mathsf{Mod}}
\DeclareMathOperator{\Fld}{\mathsf{Fld}}
\DeclareMathOperator{\Top}{\mathsf{Top}}
\DeclareMathOperator{\Vect}{\mathsf{Vect}}
\DeclareMathOperator{\Sets}{\mathsf{Set}}
\DeclareMathOperator{\Obj}{Obj}
\DeclareMathOperator{\Hom}{Hom}
\DeclareMathOperator{\Mor}{Mor}
\DeclareMathOperator{\End}{End}
\DeclareMathOperator{\Aut}{Aut}
\DeclareMathOperator{\Inn}{Inn}
\DeclareMathOperator{\Syl}{Syl}
\DeclareMathOperator{\Gal}{Gal}
\DeclareMathOperator{\im}{im}
\DeclareMathOperator{\id}{id}

\DeclareMathOperator{\coker}{coker}
\DeclareMathOperator{\spn}{span}
\DeclareMathOperator{\Map}{Map}
\DeclareMathOperator{\fin}{fin}
\DeclareMathOperator{\irr}{irr}
\DeclareMathOperator{\Hess}{Hess}
\DeclareMathOperator{\vol}{vol}
\DeclareMathOperator{\Alt}{Alt}
\DeclareMathOperator{\Mat}{Mat}
\DeclareMathOperator{\Sym}{Sym}
\DeclareMathOperator{\ord}{ord}
\DeclareMathOperator{\sgn}{sgn}
\DeclareMathOperator{\Ind}{Ind}
\DeclareMathOperator{\Stab}{Stab}
\DeclareMathOperator{\Char}{Char}
\DeclareMathOperator{\dom}{dom}
\DeclareMathOperator{\cod}{cod}
\DeclareMathOperator{\sk}{sk}
\DeclareMathOperator{\tp}{Top}
\DeclareMathOperator{\Fun}{Fun}
\DeclareMathOperator{\sSet}{sSet}
\DeclareMathOperator{\orb}{Orb}
\DeclareMathOperator{\Div}{Div}
\DeclareMathOperator{\res}{Res}
\DeclareMathOperator{\Cone}{\mathsf{Cone}}
\DeclareMathOperator{\nl}{null}
\DeclareMathOperator{\Tor}{Tor}
\DeclareMathOperator{\tors}{Tors}
\DeclareMathOperator{\spec}{Spec}
\DeclareMathOperator{\colim}{Colim}

% Section formatting
\titleformat{\section}
    {\Large\sffamily\scshape\bfseries}{\thesection}{1em}{}
\titleformat{\subsection}[runin]
    {\large\sffamily\bfseries}{\thesubsection}{1em}{}
\titleformat{\subsubsection}[runin]{\normalfont\itshape}{\thesubsubsection}{1em}{}

\title{Algebraic Topology I}
\author{Lectures by Andrew Blumberg, Notes by Yousif Elhag}
\date{SEMESTER}

\newcommand*{\titleSW}
    {\begingroup% Story of Writing
    \raggedleft
    \vspace*{\baselineskip}
    {\Huge\itshape Algebraic Topology 1}\\[\baselineskip]
    {\large\itshape Notes by Yousif Elhag}\\[0.2\textheight]
    {\Large Lectures by Andrew Blumberg}\par
    \vfill
    {\Large \sffamily Columbia University}
    \vspace*{\baselineskip}
\endgroup}
\pagestyle{simple}

\chapterstyle{ell}


%\renewcommand{\cftchapterpagefont}{}
\renewcommand\cftchapterfont{\sffamily}
\renewcommand\cftsectionfont{\scshape}
\renewcommand*{\cftchapterleader}{}
\renewcommand*{\cftsectionleader}{}
\renewcommand*{\cftsubsectionleader}{}
\renewcommand*{\cftchapterformatpnum}[1]{~\textbullet~#1}
\renewcommand*{\cftsectionformatpnum}[1]{~\textbullet~#1}
\renewcommand*{\cftsubsectionformatpnum}[1]{~\textbullet~#1}
\renewcommand{\cftchapterafterpnum}{\cftparfillskip}
\renewcommand{\cftsectionafterpnum}{\cftparfillskip}
\renewcommand{\cftsubsectionafterpnum}{\cftparfillskip}
\setrmarg{3.55em plus 1fil}
\setsecnumdepth{subsection}
\maxsecnumdepth{subsection}
\settocdepth{subsection}

\begin{document}
    
\begin{titlingpage}
\titleSW
\end{titlingpage}

\thispagestyle{empty}



\tableofcontents

\chapter{Learning the Lingo}% \label{cha:basic_notions}
The following is a transcription of notes for Professor Andrew Blumberg's ``Algebraic Topology 1." For consultation regarding the material in the notes, his office is room 607 in the Math department. Although there isn't a strict textbook this course will be using, some choice texts to read ahead of the notes will be 
\begin{itemize}
    \item Peter May's ``Concise course in Algebraic Topology"
    \item Haynes Miller's ``Notes on Algebraic Topology"
    \item Munkres's ``Elements of Algebraic Topology"
    \item Weibel's ``Homological Algebra"
    \item Saunders \& Maclane's ``Categories for the Working Mathematician" (Although, IMO, Riehl's ``Category Theory in Context" is better written.)
\end{itemize}
Algebraic topology, at its core, answers questions regarding classification, turning geometric problems into ones relying on algebra (which is somehow supposed to be easier). For instance, consider the spaces $\R^2$ and $\R^3$; 
\par \underline{Question:} Are $\R^2$ and $\R^3$ the same as sets? $\rightarrow$ Well, although there is a bijection $\R^3 \iso \R^2$, this is insufficient for the ways we want to think about things in Algebraic topology. As we'll later see, $\R^2 \neq \R^3$ in our senses of ``being the same."
\section{Category Theory for People who aren't Peter May}% \label{sec:categorical_notions}
\begin{defn}
    A \emph{category} $C$ is a collection of data consisting of objects $\Obj(C)$ and a collection of morphisms between said-objects. Each object $X \in \Obj(C)$ has an identity morphism $1_X: X \to X$. Additionally, any morphisms $f,g \in \Mor(C)$ are associative with respect to composition.
\end{defn}

\begin{exm}
    One useful example of a category we may work with is the category $\Vect$, with objects consisting vector spaces with linear maps as morphisms. 
\end{exm}

\begin{defn}
    Given two categories $C, D$, then a \emph{functor} $F: C \to D$ is a map of categories such that: 
    \begin{enumerate}
        \item $F$ takes objects in $C$ to objects in $D$, ie. $x \in \Obj(C) \mapsto F(x) \in \Obj(D)$.
        \item For an object $X \in \Obj(C)$, we have functors acting on morphisms $f \in \Mor(C)$, ie. $f(X) \in \Obj(C) \mapsto F(f(X)) \in \Obj(D)$.
    \end{enumerate}
\end{defn}
One useful aspect of category theory is that we can give definitions that better specialize familiar notions:

\begin{defn}
    In a category $C$, we call a morphism $f: X \to Y$ an \emph{isomorphism} if there exists a map $g: Y \to X$ such that $f \circ g \iso \id_Y$ and $g \circ f \iso \id_X$. 
\end{defn}
\begin{exm*}
    For instance, for the category $C$ given by the diagram:
    \[\begin{tikzcd}
        \bullet_0 \arrow[r] \arrow[d] & \bullet_1 \\ \bullet_2
    \end{tikzcd}\]
    We see that the only isomorphisms are the identity morphisms of each object. In the category $\Top$, the isomorphisms are homeomorphisms. In the category $\Vect$, the isomorphisms are are invertible linear maps of vector spaces.
\end{exm*}

\begin{addm*}
    As noted in class, functors of categories $F: C \to D$ preserve isomorphisms, ie. $F(f) \circ F(g) = F(f \circ g) = F(id_Y)= \id_{F(X)}$.
\end{addm*}
\par 
Now, if we were to be working in the category $\Sets$, then taking $X, Y \in \Obj(\Sets)$, we define cartesian products as $X \times Y  := \{ (a,b) \mid a \in X, b \in Y\}$. However, this definition of products only really works when we're in the category $\Sets$, and doesn't bode well in general. As such, we have to work to adapt this definition more broadly in categorical language. 

\begin{defn}
    Given maps $f: A \to X$ and $g: A \to Y$, observe that we obtain a unique product map $A \to X \times Y$ given by $x \mapsto (f(x), g(x))$. Using this, we define a categorical \emph{product}, as saying that anytime we have maps $A \to X$ and $A \to Y$, we have a unique map $A \to X \times Y$, giving the diagram:
    \[\begin{tikzcd}[row sep = large]
A \arrow[drr, bend left] \arrow[ddr, bend right] \arrow[dr, dotted, description] & & \\ & X \times Y \arrow[r, "\pi_1"] \arrow[d, "\pi_2"] & X \arrow[d] \\ & Y \arrow[r] & X
\end{tikzcd}\]
\end{defn}

\begin{defn}
    The dual-notion of a product, or \emph{coproduct}, is built up analogously. Given maps $Y \to Z$ and $X \to Z$, we obtain a colimit by ``flipping"\footnote{Slightly imprecise language, we are considering maps from the empty sets into $X, Y \in \Obj(C)$ as well and using those to construct our notion of coproducts} the morphisms of the above diagram so as to obtain the diagram:
    \[ \begin{tikzcd}[row sep=huge]
        & Z  & \\
        X\ar[ur] \ar[r,"\pi_{X}", swap] & X \coprod Y \ar[u,dashed] & Y \ar[ul] \ar[l,"\pi_{Y}"] \\ & \varnothing \arrow[ur] \arrow[ul]&
    \end{tikzcd}
    \]
\end{defn}

\section{More Category Theory for People Who Aren't Peter May}

As a side-note, I'll be noting some additional things not covered in the lecture since I feel these are important pre-requisites to include for my own full understanding of the material. I'll try my best not to obfuscate Blumberg's exposition of the material.

\begin{defn}
For a category $C$, the \emph{opposite category} $C\opp$ is a category including the following data:
\begin{itemize}
    \item The objects of $C^{\text{op}}$ are the same as those in $C$.
    \item The morphisms $f\opp \in C\opp$ switches the domains and codomains for each morphism in $f \in \Mor(C)$, ie. 
    \[[f\opp : Y \to X] \in \Mor(C\opp)  \leftrightsquigarrow [f: X \to Y] \in \Mor(C)\]
\end{itemize}
\end{defn}


\begin{lem}
    In a category $C$, the following are equivalent: 
    \begin{enumerate}
        \item $f \in \Map_C(x,y)$ is an isomorphism.
        \item For all objects $z \in \Obj(C)$, post-composition with $f: x \to y$ defines a bijection.
        \[\Map_C(z,x) \xrightarrow{\iso} \Map_C(z,y)\]
        \item For all objects $d \in \Obj(C)$, pre-composition with $f: x \to y$ defines a bijection 
        \[\Map_C(y,d) \xrightarrow{\iso} \Map_C(x,d)\]
    \end{enumerate}
\end{lem}
\begin{proof}
    Assuming (1.), then $f: x \to y$ has an inverse $g: y \to x$ where, by associativity and identity laws for composition over the category $C$, post-composition defines an inverse function 
    \[g_\ast: \Map_C(z,y) \to \Map_C(z,x)\]
    to the function $f_\ast: \Map_C(z,x) \to \Map_C(z,y)$. As we can see, $g_\ast \circ f_\ast: \Map_C(z,x) \to \Map_C(z,x)$ and $f_\ast \circ g_\ast: \Map_C(z,y) \to \Map_C(z,y)$ are both identity functions. Now, assuming (ii.), there must be $g\in \Map_C(y,x)$ whose image under image under $f_\ast: \Map_C(y,x) \to \Map_C(y,y)$ is the identity $1_y$. By construction, $1_y = f\circ  g$, but by associativity, the elements of $gf$, $1_x \in \Map_C(x,x)$ have the common image $f$ under the function $f_\ast: \Map_C(x,x) \to \Map_C(x,y)$ when $gf = 1_x$. Therefore, $(1.) \iff (2.)$; We obtain $(1.) \iff (3.)$ by duality.
\end{proof}

\subsection{Sidenote: Enriched Categories}
Let us consider the category $\Top$, with which we know $\Map_C(x,y)$ is itself an object of $\Top$. We express an interest in categories $C$ where $\Map_C(x,y)$ is an abelian group. To better capture this notion, we introduce the idea of an enriched category.

We consider enriched categories as categories where hom-sets (ie. $\Map_C(x,y)$) are not just sets and have more structure in an enriched category $\mathcal{V}$.
\begin{addm*}
    Consider three objects $U,V,W \in \Obj(C)$. Given morphisms $f \in \Map_C(U,V)$ and $g \in \Map_C(V,W)$, we know that there exists a composition map $g \circ f \in \Map_C(U,W)$. Viewing all objects of the form $\Map_C(-,-)$ as objects of an enriched category $\mathcal{V}$, we wish to have a relation of the form 
\[\Map_C(U,V) \times \Map_C(V,W) \to \Map_C(U,W)\]
One of the conditions to have this structure on $\mathcal{V}$ is to induce the structure of a \textcolor{red}{monoidal category}.
\end{addm*}

\newpage
\begin{defn} 
    A {\emph{monoidal category}} $(\mathcal{V}, \otimes, i)$ consists of the following data:
    \begin{enumerate}
        \item A category $\mathcal{V}$
        \item A monoidal product as a bi-functor $\otimes: \mathcal{V} \times \mathcal{V} \to \mathcal{V}$
        \item A monoidal unit $i$
    \end{enumerate}
    which satisfies the natural isomorphisms expressing associativity and unitality of the monoidal product, ie. 
    \[\alpha: X \otimes (Y \otimes Z) \xrightarrow{\iso} (X \otimes Y) \otimes Z, \ \ \lambda: i \otimes X \xrightarrow{\iso} X,  \ \ \rho: X \otimes i \xrightarrow{\iso} X\]
\end{defn}

\begin{addm}
    A \emph{symmetric monoidal category} has the additional condition that the monoidal product is commutative, meaning that there is an additional natural isomorphism $X \otimes Y \iso Y \otimes X$
\end{addm}

\begin{exm}
    The triple $(\Mod_R, \otimes_R, R)$ where $\Mod_R$ is the category of modules over a commutative ring $R$ is a monoidal category.
\end{exm}

\begin{defn}
    A \emph{category $C$ enriched over $\mathcal{V}$ (or a $\mathcal{V}$-category $\bold{C}$)} is given by the data of:
    \begin{enumerate}
        \item A collection of objects, denoted $\Obj(C)$.
        \item For each ordered pair of objects $X,Y \in \Obj(C)$, there is an object $\Map_C(X,Y) = \bold{C}(X,Y)\ \text{in } \mathcal{V}$.
        \item For each ordered triple $X,Y,Z \in C$, there is a morphism $\circ: \bold{C}(X,Y) \otimes \bold{C}(Y,Z) \to \bold{C}(X,Z)$ in $\mathcal{V}$.
        \item For each object $X \in \Obj(C)$, there is a morphism $\id_X: i \to \bold{C}(X,X)$ in $\mathcal{V}$ such that 
        \begin{itemize}
            \item For each $W,X,Y,Z \in \Obj(C)$, the composition in $\bold{C}$ is associative such that the diagram:
            \[\begin{tikzcd}
                \bold{C}(Y,Z) \otimes \bold{C}(X,Y) \otimes \bold{C}(W,X) \ar[r, "1 \otimes \circ"] \ar[d, "\circ \otimes 1"] & \bold{C}(Y,Z) \otimes \bold{C}(W,Y)  \ar[d, "\circ"] \\
                \bold{C}(X,Z) \otimes \bold{C}(W,X) \ar[r, "\circ"] & \bold{C}(W,Z)
            \end{tikzcd}\]
           is commuting.
           \item For each $X,Y,Z \in \Obj(C)$, the following diagram commutes:
           \[\begin{tikzcd}
               \bold{C}(X,Y) \otimes i \ar[d, "1\otimes \id_X"] \ar[dr, "\iso"] &  &  i \otimes \bold{C}(X,Y) \ar[dl, "\iso", swap] \ar[d, "\id_Y\otimes 1"] 
               \\
               \bold{C}(X,Y) \otimes \bold{C}(X,X) \ar[r, "\circ"] & \bold{C}(X,Y) & \ar[l, "\circ"]\bold{C}(Y,Y) \otimes \bold{C}(X,Y)
           \end{tikzcd}\]
        \end{itemize}
    \end{enumerate}
\end{defn}

\begin{exm}[Enriched Categories] We list some examples of enriched categories:
    \begin{itemize}
        \item Given the symmetric monoidal category $\mathcal{V} = (\Vect_K, \otimes_K, K)$, we can define $\Vect_K$ to be a $\mathcal{V}$-category. For two linear maps $f,g\in \Vect_K(U,W)$, we can easily verify and check that we have associativity and commutativity as in (Definition 1.2.6).
        \item Consider the category of modules over a fixed commutative ring $R$, which we denote $\Mod_R$. The category of Abelian groups $\Ab$ has a monoidal structure such that $\Mod_R$ is enriched over $(\Ab, \otimes_{\Z}, \Z)$.
    \end{itemize}
\end{exm}

\newpage
\subsection{Natural Transformations \& (Co)Limits}

\begin{defn}
    Given categories $C$ and $D$, and functors $F,G: C \rightrightarrows D$, a \emph{natural transformation} $\alpha: F \implies G$ consists of:
    \begin{enumerate}
        \item An arrow $\alpha_C: F(c) \to G(c)$ for each object $c \in \Obj(C)$, the collection of which defines the components of the natural transformation.
        \item For each morphism $f: c \to d$ in $C$, we have a commuting diagram:
        \[\begin{tikzcd}
            F(c) \ar[r, "\alpha_C"] \ar[d, "F(f)", swap] & G(c) \ar[d, "G(f)"] \\
            F(d) \ar[r, "\alpha_D"] & G(d)
        \end{tikzcd}\]
    \end{enumerate}
\end{defn}
\begin{defn}
    A \emph{natural isomorphism} is a natural transformation $\alpha: F \implies G$ in which every component $\alpha_C$ is an isomorphism.
\end{defn}


\begin{defn}
    For any object $c \in \Obj(C)$ and any category $J$, the \emph{constant functor} $I_C: J \to C$ is a functor sending every object of $J$ to $c \in C$ and every morphism $f \in \Mor(J)$ to the identity morphism $1_c$.
\end{defn}

\begin{defn}
    A \emph{cone} over a diagram $F: J \to C$ with summit $c \in \Obj(C)$ is a natural transformation $\lambda: I_c \implies F$ whose domain is the constant functor at $c$. The components $\{\lambda_i: c \to F_i\}_{i \in I}$ of the natural transformation are called the \emph{legs of the cone}. More explicity,
    \begin{itemize}
        \item The data of a cone $F: J \to C$ with summit $c$ consists of a collection of morphisms $\lambda_i: c \to F_i$. 
        \item A family of morphisms $\{\lambda_i: c \to F_i\}$ defines a cone over $F$ iff for each morphism $f: x \to y$ in $J$, the following triangle commutes in $C$:
        \[\begin{tikzcd}
            & c \ar[dl, "\lambda_x", swap] \ar[dr, "\lambda_y"]& \\
            Fx \ar[rr, "F(f)", swap] &  & Fy
        \end{tikzcd}\]
    \end{itemize}
    Dually, a cone under $F$ with nadir $c \in \Obj(C)$ is a natural transformation $\lambda: F \implies I_C$ whose legs are components $\{\lambda_j: F_i\to c\}_{i \in I}$. For each morphism $f: x \to y$ in J, the above triangle with morphisms $\lambda_x, \lambda_y$ flipped will commute.
\end{defn}
  \begin{addm}
      A \emph{cone under a diagram} $F: J \to C$ is also called a cocone and is defined analogously to cones over a diagram. A cone under $F: J \to C$ is precisely a cone over $F\opp: J\opp \to C\opp$.
  \end{addm}

\begin{defn}
    For any diagram $F: J \to C$, there is a functor, 
    \[\Cone(-,F): C\opp \to \Sets\]
    which sends $c \in C$ to the set of cones over $F$ with summit $c$. Using the Yoneda Lemma\footnote{Recall that Yoneda's Lemma states: For any functor $F: C \to \Sets$ whose domain $C$ is locally small, then for any object $c \in C$, there's a bijection:
    \[\Hom(\bold{C}(c,-), F) \iso Fc\]}, a \emph{limit} consists of an object $\lim F\in C$ together with a universal cone $\lambda: \lim F \implies F$, called the limit cone, defining a natural isomorphism 
    \[\bold{C}(-,\lim F) \iso \Cone(-,F)\]
\end{defn}

\newpage
\begin{defn}
    Dually, there is a functor $\Cone(F,-): C \to \Sets$ that sends $c \in C$ to the set of cones with nadir $c$. A \emph{colimit} of $F$ is a representation for $\Cone(F,-)$. Again, by Yoneda's Lemma, a colimit consists of an object $\colim F \in C$ together with a universal conve $\lambda: F \implies \colim F$, called the colimit cone, defining a natural isomorphism:
    \[\bold{C}(\colim F, -) \iso \Cone(F, -)\]
\end{defn}

\begin{addm*}
    We may also equivalently define limits of $F$ as the terminal object in the category of cones over $F$ and colimits as the initial object in the category of cones over $F$.
\end{addm*}

\begin{exm}[Definition of Product]
    A product is a limit of a diagram indexed by a discrete category with only identity morphisms. A diagram in $C$ indexed by a discrete category $J$ consists of a collection of objects $F_j \in C$ indexed by $j \in J$.
    
    A cone over this diagram with summit $c \in C$ is a $J$-indexed family of morphisms $\{\lambda_j: c \to F_j\}_{j \in J}$. This limit is denoted by $\Pi_{j\in J} F_j$ and the legs of the cone are maps,
    \[\bigg(\pi_k: \prod_{j \in J} F_j \to F_k \bigg)_{k \in J}\]
\end{exm}

\begin{defn}
    A category is \emph{complete} if it contains all limits. A category is \emph{cocomplete} if it contains all colimits.
\end{defn}

\begin{exm}
    The category $\Sets$ is complete and cocomplete. The category $\Fun(C\opp, \Sets)$ is also complete and cocomplete; Observe that given the map $Y: C \to \Fun(C\opp, \Sets)$, which maps $x \mapsto \Map_C(-,x) = \bold{C}(-,x)$, that $Y$ is a fully-faithful\footnote{Recall that for locally small categories $C$ and $D$, the functor $F: C \to D$ induces a function $F_{X,Y}:  \Map_C(x,y) \to \Map_D(F(x), F(y))$. We say $F$ is faithful if $F_{X,Y}$ is injective, and full if $F_{X,Y}$ is surjective.} functor, and so the map 
    \[\Map_C(x,y) \xrightarrow{\iso} \Map(Y(x), Y(y))\]
    is an isomorphism.
\end{exm}

\subsubsection{Why does this matter?}
Consider the category $\Top$. When constructing objects like Klein bottles, $\R$ or $\C$ projective space, or any cell complex via attaching maps, we define these as sets equipped with particular topologies. 

These topologies can all be uniformly defined by our notions of limits and colimits via universal cones. As we'll observe, constructed topological spaces can be characterized either as a limit or colimit of a specific diagram over the category $\Top$. By mapping things out of standard topological objects (For instance, mapping out of $S^n$, ie. $\{\Map_C(S^n, -)\}$), we're able to extrapolate more information about these objects.

$\Map_{\Top}(X,Y)$ is a space. Considering the map $H: [0,1] \to \Map_{\Top}(X,Y)$ with path homotopies given between $f:X \to Y \leftrightsquigarrow H(0)$ and $g: X \to Y \leftrightsquigarrow H(1)$, we'll later see that these are the same.  The maps $\{\Map_C(S^n, -)\}$ are the same if there exists a path between them, and so we develop a notion of \textcolor{red}{homotopy} via $\Map_{\Top}(I, \Map_{\Top}(X,Y))$.








\end{document}.